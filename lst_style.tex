\renewcommand{\lstlistingname}{代码}


% define some basic colors
\definecolor{mauve}{rgb}{0.58,0,0.82}


\setmonofont{Consolas}

\lstset{
% listings sonderzeichen (for german weirdness)
literate={ö}{{\"o}}1
{ä}{{\"a}}1
{ü}{{\"u}}1,
basicstyle=\tiny\ttfamily,                    % very small code
breaklines=true,                              % break long lines
commentstyle=\itshape\color{green!50!black},  % comments are green
keywordstyle=[1]\color{blue!80!black},        % instructions are blue
keywordstyle=[2]\color{orange!80!black},      % sections/other directives are orange
keywordstyle=[3]\color{red!50!black},         % registers are red
stringstyle=\color{mauve},                    % strings are from the telekom
identifierstyle=\color{teal},                 % user declared addresses are teal
frame=l,                                      % black line on the left side of code
language=[RISC-V]Assembler,                   % all code is RISC-V
tabsize=4,                                    % indent tabs with 4 spaces
showstringspaces=false                        % do not replace spaces with weird underlines
}

\lstdefinestyle{mystyle}
{
    backgroundcolor=\color{backcolour},
    commentstyle=\color{codegreen},
    keywordstyle=\color{magenta},
    numberstyle=\tiny\color{codegray},
    stringstyle=\color{codepurple},
    basicstyle=\ttfamily\footnotesize,
    breakatwhitespace=false,
    breaklines=true,
    captionpos=b,
    keepspaces=true,
    numbers=none,
    numbersep=5pt,
    showspaces=false,
    showstringspaces=false,
    showtabs=false,
    tabsize=2,
    frame=none
}

\lstset{style=mystyle,language=C++}
