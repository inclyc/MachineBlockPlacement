2011 年,基于分支概率的基本块被引入,Branch Probability Basic Block Placement 第一次全名出现在了 LLVM 中 \cite{llvmmachineblockplacementintro2011}.这个算法先把所有的 BB 形成若干链,然后基于强连通分量合并这些链,形成最终的layout.

这个算法很快被废弃,在 11 月完成了现在所看的版本的雏形,同时也是上面介绍给大家的版本 \cite{llvmmachineblockplacementimporve2011-1, llvmmachineblockplacementimporve2011-2}. 这个版本的算法递归地从循环中构建链,同时尽量保证 CFG 的拓扑序正确.

\begin{table}
    \begin{tabular}{ccc}
        \toprule
        版本                                            & 如何处理 Loop                                                         & 如何从 Chain 到最终布局 \\
        \midrule
        \cite{llvmmachineblockplacementintro2011}     & 无特殊处理                                                             & 强连通分量排序         \\
        \cite{llvmmachineblockplacementimporve2011-1} & 递归建立 Loop-Based Chains                                            & 逆后序遍历           \\
        \cite{llvmmachineblockplacementimporve2011-2} & 基于版本 \cite{llvmmachineblockplacementimporve2011-1} 多建立一个 WorkList & 函数整体看成一个Chain   \\
        \bottomrule
    \end{tabular}
    \caption{三个实现的区别}
\end{table}